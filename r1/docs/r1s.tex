\documentclass [11pt] {jsarticle}

\begin {document}
\section*{ソースコード}
\begin{verbatim}[4]
   val id = shapeGroup.selectedToggle().
   asInstanceOf[javafx.scene.control.ToggleButton].id()
   
#include <stdio.h>
#include <math.h>
//問題1
int calc(float x0, int ind) {
	float x=0;
	int i;
	for (i=0; i<pow(10,ind); i++) {
		x += x0;
	}
	printf("n=10^%d: %f\n",ind , x);
	return 0;
}

int main() {
	float x1 = 1;
	float x2 = 0.1;
	float x3 = 0.15625;
	
	printf("x=1:\n");
	calc(x1,2);
	calc(x1,5);
	calc(x1,8);
	printf("\nx=0.1:\n");
	calc(x2,2);
	calc(x2,5);
	calc(x2,8);
	printf("\nx=0.15625:\n");
	calc(x3,2);
	calc(x3,5);
	calc(x3,8);
	return 0;
}
\end{verbatim}

問題1のfloat型を用いたもの。double型にするにはfloatをすべてdoubleに書き換えればよい。
\begin{verbatim}
#include <stdio.h>
#include <math.h>
//問題1
int calc(float x0, int ind) {
	float x=0;
	int i;
	for (i=0; i<pow(10,ind); i++) {
		x += x0;
	}
	printf("n=10^%d: %f\n",ind , x);
	return 0;
}

int main() {
	float x1 = 1;
	float x2 = 0.1;
	float x3 = 0.15625;
	
	printf("x=1:\n");
	calc(x1,2);
	calc(x1,5);
	calc(x1,8);
	printf("\nx=0.1:\n");
	calc(x2,2);
	calc(x2,5);
	calc(x2,8);
	printf("\nx=0.15625:\n");
	calc(x3,2);
	calc(x3,5);
	calc(x3,8);
	return 0;
}
\end{verbatim}

問題2のソースコード
\begin{verbatim}
#include <stdio.h>
#include <math.h>
//問題2
int calc(int n) {
	float x = 0;
	float x0 = powf(10,-n);
	int i;
	for (i=0; i<pow(10,n); i++) {
		x += x0;
	}
	printf("n=%d: %f\n", n, x);
	return 0;
}

int main() {
	int j;
	for (j=1; j<10; j++){
		calc(j);
	}
	return 0;
}
\end{verbatim}

問題3のソースコード
\begin{verbatim}
#include <stdio.h>
#include <math.h>
//問題3
int calc1(int n) {
	int x;
	x = powf(2, n);
	
	printf("n=%d: %d\n", n, x);
	return 0;
}

int calc2(int n) {
	float y;
	y = powf(2, -n);
	
	printf("n=%d: %e\n", n, y);
	return 0;
}

int main() {
	int j;
	
	for (j=1; j<=256; j++){
		calc1(j);
	}
	printf("\n");
	
	for (j=1; j<=256; j++){
		calc2(j);
	}
	return 0;
}
\end{verbatim}
\end {document}

\maketitle